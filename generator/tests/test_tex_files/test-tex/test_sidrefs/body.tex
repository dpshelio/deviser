% -*- TeX-master: "main"; fill-column: 72 -*-
\section{Package syntax and semantics}

In this section, we define the syntax and semantics of the
\SIDReferencesPackage for \sbmlthreecore. We expound on the various data
types and constructs defined in this package, then in \sec{examples}, we
provide complete examples of using the constructs in example SBML
models.

\subsection{Namespace URI and other declarations necessary for using
this package}
\label{xml-namespace}

Every SBML Level~3 package is identified uniquely by an XML namespace
URI. For an SBML document to be able to use a given SBML Level~3
package, it must declare the use of that package by referencing its URI.
The following is the namespace URI for this version of the
\SIDReferencesPackage for SBML Level~3 Version~1:

\begin{center}
\uri{http://www.sbml.org/sbml/level3/version1/refs/version1}
\end{center}

In addition, SBML documents using a given package must indicate whether
understanding the package is required for complete mathematical
interpretation of a model, or whether the package is optional. This is
done using the attribute \token{required} on the \token{<sbml>} element
in the SBML document. For the \SIDReferencesPackage the value of the
required attribute is \val{false}.

% \begin{figure}[ht!]
%   \centering
%   \includegraphics[width=0.9\textwidth]{figures/refs_version_1_complete.png}\\
% \caption{A UML representation of the \RefsPackage. See
% \ref{conventions} for conventions related to this figure. }
%   \label{fig:refs_version_1_complete}
% \end{figure}

\subsection{Primitive data types}
\label{primitive-types}

Section~3.1 of the SBML Level~3 specification defines a number of
primitive data types and also uses a number of XML Schema 1.0 data types
\citep{biron:2000}. We assume and use some of them in the rest of this
specification, specifically \primtype{boolean}, \primtype{ID},
\primtype{SId}, \primtype{SIdRef}, and \primtype{string}. The
\SIDReferences Package defines other primitive types; these are
described below.

\TODO{check all necessary types from core are listed}

% ---------------------------------------------------------
\subsection{The extended \class{Model} class}
\label{extended-model-class}

% \begin{figure}[ht!]
%   \centering
%   \includegraphics[scale=0.6]{figures/refs_extended_model_uml.pdf}\\
% \caption{A UML representation of the extended \Model class for the
% \RefsPackage. See \ref{conventions} for conventions related to this
% figure. }
%   \label{fig:refs_extended_model_uml}
% \end{figure}


\TODO{explain where Model comes from}

The \SIDReferencesPackage extends the \class{Model} object with the
addition of
a \ThingA object
, a \ThingB object
and a \ListOfThingCs object.

% ---------------------------------------------------------
\subsection{The \class{ThingA} class}
\label{thinga-class}

% \begin{figure}[ht!]
%   \centering
%   \includegraphics[scale=0.6]{figures/refs_thinga_uml.pdf}\\
% \caption{A UML representation of the \ThingA class for the
% \RefsPackage. See \ref{conventions} for conventions related to this
% figure. }
%   \label{fig:refs_thinga_uml}
% \end{figure}


\TODO{explain ThingA}

The \ThingA object derives from the \SBase class and thus inherits any
attributes and elements that are present on this class.
In addition the \ThingA object has the following attributes.

\paragraph{The \fixttspace\token{id} attribute}

A \ThingA has a required attribute \token{id} of type \primtype{SId}.
\TODO{explain id}


% ---------------------------------------------------------
\subsection{The \class{ThingB} class}
\label{thingb-class}

% \begin{figure}[ht!]
%   \centering
%   \includegraphics[scale=0.6]{figures/refs_thingb_uml.pdf}\\
% \caption{A UML representation of the \ThingB class for the
% \RefsPackage. See \ref{conventions} for conventions related to this
% figure. }
%   \label{fig:refs_thingb_uml}
% \end{figure}


\TODO{explain ThingB}

The \ThingB object derives from the \SBase class and thus inherits any
attributes and elements that are present on this class.
In addition the \ThingB object has the following attributes.

\paragraph{The \fixttspace\token{id} attribute}

A \ThingB has a required attribute \token{id} of type \primtype{SId}.
\TODO{explain id}


% ---------------------------------------------------------
\subsection{The \class{ListOfThingCs} class}
\label{listofthingcs-class}

\TODO{explain ListOfThingCs}

The \ListOfThingCs object derives from the \class{SBase} and inherits
the core attributes and subobjects from that class. It contains one or
more objects of type \ThingC.

% ---------------------------------------------------------
\subsection{The \class{ThingC} class}
\label{thingc-class}

% \begin{figure}[ht!]
%   \centering
%   \includegraphics[scale=0.6]{figures/refs_thingc_uml.pdf}\\
% \caption{A UML representation of the \ThingC class for the
% \RefsPackage. See \ref{conventions} for conventions related to this
% figure. }
%   \label{fig:refs_thingc_uml}
% \end{figure}


\TODO{explain ThingC}

The \ThingC object derives from the \SBase class and thus inherits any
attributes and elements that are present on this class.
In addition the \ThingC object has the following attributes.

\paragraph{The \fixttspace\token{thing} attribute}

A \ThingC has a required attribute \token{thing} of type
\primtype{SIdRef}.
This attribute must be the identifier of an existing \ThingA or \ThingB
object.
\TODO{explain thing}


\paragraph{The \fixttspace\token{id} attribute}

A \ThingC has a required attribute \token{id} of type \primtype{SId}.
\TODO{explain id}


