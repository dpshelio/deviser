% -*- TeX-master: "main"; fill-column: 72 -*-
\section{Package syntax and semantics}

In this section, we define the syntax and semantics of the
\GroupsPackage for \sbmlthreecore. We expound on the various data types
and constructs defined in this package, then in \sec{examples}, we
provide complete examples of using the constructs in example SBML
models.

\subsection{Namespace URI and other declarations necessary for using
this package}
\label{xml-namespace}

Every SBML Level~3 package is identified uniquely by an XML namespace
URI. For an SBML document to be able to use a given SBML Level~3
package, it must declare the use of that package by referencing its URI.
The following is the namespace URI for this version of the
\GroupsPackage for SBML Level~3 Version~1:

\begin{center}
\uri{http://www.sbml.org/sbml/level3/version1/groups/version1}
\end{center}

In addition, SBML documents using a given package must indicate whether
understanding the package is required for complete mathematical
interpretation of a model, or whether the package is optional. This is
done using the attribute \token{required} on the \token{<sbml>} element
in the SBML document. For the \GroupsPackage the value of the required
attribute is \val{false}.

% \begin{figure}[ht!]
%   \centering
%   \includegraphics[width=0.9\textwidth]{figures/groups_version_1_complete.png}\\
% \caption{A UML representation of the \GroupsPackage. See
% \ref{conventions} for conventions related to this figure. }
%   \label{fig:groups_version_1_complete}
% \end{figure}

\subsection{Primitive data types}
\label{primitive-types}

Section~3.1 of the SBML Level~3 specification defines a number of
primitive data types and also uses a number of XML Schema 1.0 data types
\citep{biron:2000}. We assume and use some of them in the rest of this
specification, specifically \primtype{boolean}, \primtype{ID},
\primtype{SId}, \primtype{SIdRef}, and \primtype{string}. The \Groups
Package defines other primitive types; these are described below.

\TODO{check all necessary types from core are listed}

\subsubsection{Type \fixttspace\primtypeNC{GroupKind}}

% \begin{figure}[ht!]
%   \centering
%   \includegraphics[scale=0.6]{figures/groups_type_enum_groupkind_uml.pdf}\\
% \caption{A UML representation of the \GroupKind type for the
% \GroupsPackage. See \ref{conventions} for conventions related to this
% figure. }
%   \label{fig:groups_type_enum_groupkind_uml}
% \end{figure}


The \primtype{GroupKind} is an emueration of values used to ...
\TODO{Explain use of GroupKind}

The possible values are \const{classification}, \const{partonomy} and
\const{collection}.

% ---------------------------------------------------------
\subsection{The extended \class{Model} class}
\label{model-class}

% \begin{figure}[ht!]
%   \centering
%   \includegraphics[scale=0.6]{figures/groups_extended_model_uml.pdf}\\
% \caption{A UML representation of the extended \Model class for the
% \GroupsPackage. See \ref{conventions} for conventions related to this
% figure. }
%   \label{fig:groups_extended_model_uml}
% \end{figure}


\TODO{explain where Model comes from}

The \GroupsPackage extends the \class{Model} object with the addition of
a \ListOfGroups object.

% ---------------------------------------------------------
\subsection{The \class{ListOfGroups} class}
\label{listofgroups-class}

\TODO{explain ListOfGroups}

The \ListOfGroups object derives from the \class{SBase} and inherits the
core attributes and subobjects from that class. It contains one or more
objects of type \Group.

% ---------------------------------------------------------
\subsection{The \class{Group} class}
\label{group-class}

% \begin{figure}[ht!]
%   \centering
%   \includegraphics[scale=0.6]{figures/groups_group_uml.pdf}\\
% \caption{A UML representation of the \Group class for the
% \GroupsPackage. See \ref{conventions} for conventions related to this
% figure. }
%   \label{fig:groups_group_uml}
% \end{figure}


\TODO{explain Group}

The \Group object derives from the \SBase class and thus inherits any
attributes and elements that are present on this class.
A \Group contains exactly one \ListOfMembers element.
In addition the \Group object has the following attributes.

\paragraph{The \fixttspace\token{kind} attribute}

A \Group has a required attribute \token{kind} of type
\primtype{GroupKind}.
\TODO{explain kind}


\paragraph{The \fixttspace\token{id} attribute}

A \Group has an optional attribute \token{id} of type \primtype{string}.
\TODO{explain id}


\paragraph{The \fixttspace\token{name} attribute}

A \Group has an optional attribute \token{name} of type
\primtype{string}.
\TODO{explain name}


% ---------------------------------------------------------
\subsection{The \class{ListOfMembers} class}
\label{listofmembers-class}

\TODO{explain ListOfMembers}

The \ListOfMembers object derives from the \class{SBase} and inherits
the core attributes and subobjects from that class. It contains one or
more objects of type \Member.

% ---------------------------------------------------------
\subsection{The \class{Member} class}
\label{member-class}

% \begin{figure}[ht!]
%   \centering
%   \includegraphics[scale=0.6]{figures/groups_member_uml.pdf}\\
% \caption{A UML representation of the \Member class for the
% \GroupsPackage. See \ref{conventions} for conventions related to this
% figure. }
%   \label{fig:groups_member_uml}
% \end{figure}


\TODO{explain Member}

The \Member object derives from the \SBase class and thus inherits any
attributes and elements that are present on this class.
In addition the \Member object has the following attributes.

\paragraph{The \fixttspace\token{id} attribute}

A \Member has an optional attribute \token{id} of type
\primtype{string}.
\TODO{explain id}


\paragraph{The \fixttspace\token{name} attribute}

A \Member has an optional attribute \token{name} of type
\primtype{string}.
\TODO{explain name}


\paragraph{The \fixttspace\token{idRef} attribute}

A \Member has an optional attribute \token{idRef} of type
\primtype{SIdRef}.
This attribute must be the identifier of an existing \SBase object.
\TODO{explain idRef}


\paragraph{The \fixttspace\token{metaIdRef} attribute}

A \Member has an optional attribute \token{metaIdRef} of type
\primtype{IDREF}.
\TODO{explain metaIdRef}


