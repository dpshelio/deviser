% -*- TeX-master: "main"; fill-column: 72 -*-
\section{Package syntax and semantics}

In this section, we define the syntax and semantics of the
\SpatialProcessesPackage for \sbmlthreecore. We expound on the various
data types and constructs defined in this package, then in
\sec{examples}, we provide complete examples of using the constructs in
example SBML models.

\subsection{Namespace URI and other declarations necessary for using
this package}
\label{xml-namespace}

Every SBML Level~3 package is identified uniquely by an XML namespace
URI. For an SBML document to be able to use a given SBML Level~3
package, it must declare the use of that package by referencing its URI.
The following is the namespace URI for this version of the
\SpatialProcessesPackage for SBML Level~3 Version~1:

\begin{center}
\uri{http://www.sbml.org/sbml/level3/version1/spatial/version1}
\end{center}

In addition, SBML documents using a given package must indicate whether
understanding the package is required for complete mathematical
interpretation of a model, or whether the package is optional. This is
done using the attribute \token{required} on the \token{<sbml>} element
in the SBML document. For the \SpatialProcessesPackage the value of the
required attribute is \val{true}.

% \begin{figure}[ht!]
%   \centering
%   \includegraphics[width=0.9\textwidth]{figures/spatial_version_1_complete.png}\\
% \caption{A UML representation of the \SpatialPackage. See
% \ref{conventions} for conventions related to this figure. }
%   \label{fig:spatial_version_1_complete}
% \end{figure}

\subsection{Primitive data types}
\label{primitive-types}

Section~3.1 of the SBML Level~3 specification defines a number of
primitive data types and also uses a number of XML Schema 1.0 data types
\citep{biron:2000}. We assume and use some of them in the rest of this
specification, specifically \primtype{boolean}, \primtype{ID},
\primtype{SId}, \primtype{SIdRef}, and \primtype{string}. The
\SpatialProcesses Package defines other primitive types; these are
described below.

\TODO{check all necessary types from core are listed}

\subsubsection{Type \fixttspace\primtypeNC{BoundaryConditionKind}}

% \begin{figure}[ht!]
%   \centering
%   \includegraphics[scale=0.6]{figures/spatial_type_enum_boundaryconditionkind_uml.pdf}\\
% \caption{A UML representation of the \BoundaryConditionKind type for
% the \SpatialPackage. See \ref{conventions} for conventions related to
% this figure. }
%   \label{fig:spatial_type_enum_boundaryconditionkind_uml}
% \end{figure}


The \primtype{BoundaryConditionKind} is an emueration of values used to
...
\TODO{Explain use of BoundaryConditionKind}

The possible values are \const{Robin\_valueCoefficient},
\const{Robin\_inwardNormalGradientCoefficient}, \const{Robin\_sum},
\const{Neumann} and \const{Dirichlet}.

\subsubsection{Type \fixttspace\primtypeNC{CoordinateKind}}

% \begin{figure}[ht!]
%   \centering
%   \includegraphics[scale=0.6]{figures/spatial_type_enum_coordinatekind_uml.pdf}\\
% \caption{A UML representation of the \CoordinateKind type for the
% \SpatialPackage. See \ref{conventions} for conventions related to
% this figure. }
%   \label{fig:spatial_type_enum_coordinatekind_uml}
% \end{figure}


The \primtype{CoordinateKind} is an emueration of values used to ...
\TODO{Explain use of CoordinateKind}

The possible values are \const{cartesianX}, \const{cartesianY} and
\const{cartesianZ}.

\subsubsection{Type \fixttspace\primtypeNC{DiffusionKind}}

% \begin{figure}[ht!]
%   \centering
%   \includegraphics[scale=0.6]{figures/spatial_type_enum_diffusionkind_uml.pdf}\\
% \caption{A UML representation of the \DiffusionKind type for the
% \SpatialPackage. See \ref{conventions} for conventions related to
% this figure. }
%   \label{fig:spatial_type_enum_diffusionkind_uml}
% \end{figure}


The \primtype{DiffusionKind} is an emueration of values used to ...
\TODO{Explain use of DiffusionKind}

The possible values are \const{isotropic}, \const{anisotropic} and
\const{tensor}.

\subsubsection{Type \fixttspace\primtypeNC{FunctionKind}}

% \begin{figure}[ht!]
%   \centering
%   \includegraphics[scale=0.6]{figures/spatial_type_enum_functionkind_uml.pdf}\\
% \caption{A UML representation of the \FunctionKind type for the
% \SpatialPackage. See \ref{conventions} for conventions related to
% this figure. }
%   \label{fig:spatial_type_enum_functionkind_uml}
% \end{figure}


The \primtype{FunctionKind} is an emueration of values used to ...
\TODO{Explain use of FunctionKind}

The possible values are \const{layered} and \const{layered}.

\subsubsection{Type \fixttspace\primtypeNC{GeometryKind}}

% \begin{figure}[ht!]
%   \centering
%   \includegraphics[scale=0.6]{figures/spatial_type_enum_geometrykind_uml.pdf}\\
% \caption{A UML representation of the \GeometryKind type for the
% \SpatialPackage. See \ref{conventions} for conventions related to
% this figure. }
%   \label{fig:spatial_type_enum_geometrykind_uml}
% \end{figure}


The \primtype{GeometryKind} is an emueration of values used to ...
\TODO{Explain use of GeometryKind}

The possible values are \const{cartesian} and \const{cartesian}.

\subsubsection{Type \fixttspace\primtypeNC{SetOperation}}

% \begin{figure}[ht!]
%   \centering
%   \includegraphics[scale=0.6]{figures/spatial_type_enum_setoperation_uml.pdf}\\
% \caption{A UML representation of the \SetOperation type for the
% \SpatialPackage. See \ref{conventions} for conventions related to
% this figure. }
%   \label{fig:spatial_type_enum_setoperation_uml}
% \end{figure}


The \primtype{SetOperation} is an emueration of values used to ...
\TODO{Explain use of SetOperation}

The possible values are \const{union}, \const{intersection} and
\const{relativeComplement}.

\subsubsection{Type \fixttspace\primtypeNC{InterpolationKind}}

% \begin{figure}[ht!]
%   \centering
%   \includegraphics[scale=0.6]{figures/spatial_type_enum_interpolationkind_uml.pdf}\\
% \caption{A UML representation of the \InterpolationKind type for the
% \SpatialPackage. See \ref{conventions} for conventions related to
% this figure. }
%   \label{fig:spatial_type_enum_interpolationkind_uml}
% \end{figure}


The \primtype{InterpolationKind} is an emueration of values used to ...
\TODO{Explain use of InterpolationKind}

The possible values are \const{nearestNeighbor} and \const{linear}.

\subsubsection{Type \fixttspace\primtypeNC{PolygonKind}}

% \begin{figure}[ht!]
%   \centering
%   \includegraphics[scale=0.6]{figures/spatial_type_enum_polygonkind_uml.pdf}\\
% \caption{A UML representation of the \PolygonKind type for the
% \SpatialPackage. See \ref{conventions} for conventions related to
% this figure. }
%   \label{fig:spatial_type_enum_polygonkind_uml}
% \end{figure}


The \primtype{PolygonKind} is an emueration of values used to ...
\TODO{Explain use of PolygonKind}

The possible values are \const{triangle} and \const{quadrilateral}.

\subsubsection{Type \fixttspace\primtypeNC{PrimitiveKind}}

% \begin{figure}[ht!]
%   \centering
%   \includegraphics[scale=0.6]{figures/spatial_type_enum_primitivekind_uml.pdf}\\
% \caption{A UML representation of the \PrimitiveKind type for the
% \SpatialPackage. See \ref{conventions} for conventions related to
% this figure. }
%   \label{fig:spatial_type_enum_primitivekind_uml}
% \end{figure}


The \primtype{PrimitiveKind} is an emueration of values used to ...
\TODO{Explain use of PrimitiveKind}

The possible values are \const{sphere}, \const{cube}, \const{cylinder},
\const{cone}, \const{circle} and \const{square}.

\subsubsection{Type \fixttspace\primtypeNC{DataKind}}

% \begin{figure}[ht!]
%   \centering
%   \includegraphics[scale=0.6]{figures/spatial_type_enum_datakind_uml.pdf}\\
% \caption{A UML representation of the \DataKind type for the
% \SpatialPackage. See \ref{conventions} for conventions related to
% this figure. }
%   \label{fig:spatial_type_enum_datakind_uml}
% \end{figure}


The \primtype{DataKind} is an emueration of values used to ...
\TODO{Explain use of DataKind}

The possible values are \const{double}, \const{float}, \const{uint8},
\const{uint16} and \const{uint32}.

\subsubsection{Type \fixttspace\primtypeNC{CompressionKind}}

% \begin{figure}[ht!]
%   \centering
%   \includegraphics[scale=0.6]{figures/spatial_type_enum_compressionkind_uml.pdf}\\
% \caption{A UML representation of the \CompressionKind type for the
% \SpatialPackage. See \ref{conventions} for conventions related to
% this figure. }
%   \label{fig:spatial_type_enum_compressionkind_uml}
% \end{figure}


The \primtype{CompressionKind} is an emueration of values used to ...
\TODO{Explain use of CompressionKind}

The possible values are \const{uncompressed} and \const{deflated}.

% ---------------------------------------------------------
\subsection{The extended \class{Model} class}
\label{extended-model-class}

% \begin{figure}[ht!]
%   \centering
%   \includegraphics[scale=0.6]{figures/spatial_extended_model_uml.pdf}\\
% \caption{A UML representation of the extended \Model class for the
% \SpatialPackage. See \ref{conventions} for conventions related to
% this figure. }
%   \label{fig:spatial_extended_model_uml}
% \end{figure}


\TODO{explain where Model comes from}

The \SpatialProcessesPackage extends the \class{Model} object with the
addition of
a \Geometry object.

% ---------------------------------------------------------
\subsection{The \class{Geometry} class}
\label{geometry-class}

% \begin{figure}[ht!]
%   \centering
%   \includegraphics[scale=0.6]{figures/spatial_geometry_uml.pdf}\\
% \caption{A UML representation of the \Geometry class for the
% \SpatialPackage. See \ref{conventions} for conventions related to
% this figure. }
%   \label{fig:spatial_geometry_uml}
% \end{figure}


\TODO{explain Geometry}

The \Geometry object derives from the \SBase class and thus inherits any
attributes and elements that are present on this class.
A \Geometry contains exactly one \ListOfCoordinateComponents element.
A \Geometry contains exactly one \ListOfDomainTypes element.
A \Geometry contains exactly one \ListOfDomains element.
A \Geometry contains exactly one \ListOfAdjacentDomains element.
A \Geometry contains exactly one \ListOfGeometryDefinitions element.
A \Geometry contains exactly one \ListOfSampledFields element.
In addition the \Geometry object has the following attributes.

\paragraph{The \fixttspace\token{id} attribute}

A \Geometry has an optional attribute \token{id} of type \primtype{SId}.
\TODO{explain id}


\paragraph{The \fixttspace\token{coordinateSystem} attribute}

A \Geometry has a required attribute \token{coordinateSystem} of type
\primtype{GeometryKind}.
\TODO{explain coordinateSystem}


% ---------------------------------------------------------
\subsection{The extended \class{Compartment} class}
\label{extended-compartment-class}

% \begin{figure}[ht!]
%   \centering
%   \includegraphics[scale=0.6]{figures/spatial_extended_compartment_uml.pdf}\\
% \caption{A UML representation of the extended \Compartment class for
% the \SpatialPackage. See \ref{conventions} for conventions related to
% this figure. }
%   \label{fig:spatial_extended_compartment_uml}
% \end{figure}


\TODO{explain where Compartment comes from}

The \SpatialProcessesPackage extends the \class{Compartment} object with
the addition of
a \CompartmentMapping object.

% ---------------------------------------------------------
\subsection{The \class{CompartmentMapping} class}
\label{compartmentmapping-class}

% \begin{figure}[ht!]
%   \centering
%   \includegraphics[scale=0.6]{figures/spatial_compartmentmapping_uml.pdf}\\
% \caption{A UML representation of the \CompartmentMapping class for
% the \SpatialPackage. See \ref{conventions} for conventions related to
% this figure. }
%   \label{fig:spatial_compartmentmapping_uml}
% \end{figure}


\TODO{explain CompartmentMapping}

The \CompartmentMapping object derives from the \SBase class and thus
inherits any attributes and elements that are present on this class.
In addition the \CompartmentMapping object has the following attributes.

\paragraph{The \fixttspace\token{id} attribute}

A \CompartmentMapping has a required attribute \token{id} of type
\primtype{SId}.
\TODO{explain id}


\paragraph{The \fixttspace\token{domainType} attribute}

A \CompartmentMapping has a required attribute \token{domainType} of
type \primtype{SIdRef}.
This attribute must be the identifier of an existing \DomainType object.
\TODO{explain domainType}


\paragraph{The \fixttspace\token{unitSize} attribute}

A \CompartmentMapping has a required attribute \token{unitSize} of type
\primtype{double}.
\TODO{explain unitSize}


% ---------------------------------------------------------
\subsection{The extended \class{Species} class}
\label{extended-species-class}

% \begin{figure}[ht!]
%   \centering
%   \includegraphics[scale=0.6]{figures/spatial_extended_species_uml.pdf}\\
% \caption{A UML representation of the extended \Species class for the
% \SpatialPackage. See \ref{conventions} for conventions related to
% this figure. }
%   \label{fig:spatial_extended_species_uml}
% \end{figure}


\TODO{explain where Species comes from}

The \SpatialProcessesPackage extends the \class{Species} object with the
addition of
the following attributes..

\paragraph{The \fixttspace\token{isSpatial} attribute}

A \Species has an optional attribute \token{isSpatial} of type
\primtype{bool}.
\TODO{explain isSpatial}


% ---------------------------------------------------------
\subsection{The extended \class{Parameter} class}
\label{extended-parameter-class}

% \begin{figure}[ht!]
%   \centering
%   \includegraphics[scale=0.6]{figures/spatial_extended_parameter_uml.pdf}\\
% \caption{A UML representation of the extended \Parameter class for
% the \SpatialPackage. See \ref{conventions} for conventions related to
% this figure. }
%   \label{fig:spatial_extended_parameter_uml}
% \end{figure}


\TODO{explain where Parameter comes from}

The \SpatialProcessesPackage extends the \class{Parameter} object with
the addition of
a \SpatialSymbolReference object
, an \AdvectionCoefficient object
, a \BoundaryCondition object
and a \DiffusionCoefficient object.

% ---------------------------------------------------------
\subsection{The \class{SpatialSymbolReference} class}
\label{spatialsymbolreference-class}

% \begin{figure}[ht!]
%   \centering
%   \includegraphics[scale=0.6]{figures/spatial_spatialsymbolreference_uml.pdf}\\
% \caption{A UML representation of the \SpatialSymbolReference class
% for the \SpatialPackage. See \ref{conventions} for conventions
% related to this figure. }
%   \label{fig:spatial_spatialsymbolreference_uml}
% \end{figure}


\TODO{explain SpatialSymbolReference}

The \SpatialSymbolReference object derives from the \SBase class and
thus inherits any attributes and elements that are present on this
class.
In addition the \SpatialSymbolReference object has the following
attributes.

\paragraph{The \fixttspace\token{spatialRef} attribute}

A \SpatialSymbolReference has a required attribute \token{spatialRef} of
type \primtype{SIdRef}.
This attribute must be the identifier of an existing \Geometry object.
\TODO{explain spatialRef}


% ---------------------------------------------------------
\subsection{The \class{AdvectionCoefficient} class}
\label{advectioncoefficient-class}

% \begin{figure}[ht!]
%   \centering
%   \includegraphics[scale=0.6]{figures/spatial_advectioncoefficient_uml.pdf}\\
% \caption{A UML representation of the \AdvectionCoefficient class for
% the \SpatialPackage. See \ref{conventions} for conventions related to
% this figure. }
%   \label{fig:spatial_advectioncoefficient_uml}
% \end{figure}


\TODO{explain AdvectionCoefficient}

The \AdvectionCoefficient object derives from the \SBase class and thus
inherits any attributes and elements that are present on this class.
In addition the \AdvectionCoefficient object has the following
attributes.

\paragraph{The \fixttspace\token{variable} attribute}

An \AdvectionCoefficient has a required attribute \token{variable} of
type \primtype{SIdRef}.
This attribute must be the identifier of an existing \Species object.
\TODO{explain variable}


\paragraph{The \fixttspace\token{coordinate} attribute}

An \AdvectionCoefficient has a required attribute \token{coordinate} of
type \primtype{CoordinateKind}.
\TODO{explain coordinate}


% ---------------------------------------------------------
\subsection{The \class{BoundaryCondition} class}
\label{boundarycondition-class}

% \begin{figure}[ht!]
%   \centering
%   \includegraphics[scale=0.6]{figures/spatial_boundarycondition_uml.pdf}\\
% \caption{A UML representation of the \BoundaryCondition class for the
% \SpatialPackage. See \ref{conventions} for conventions related to
% this figure. }
%   \label{fig:spatial_boundarycondition_uml}
% \end{figure}


\TODO{explain BoundaryCondition}

The \BoundaryCondition object derives from the \SBase class and thus
inherits any attributes and elements that are present on this class.
In addition the \BoundaryCondition object has the following attributes.

\paragraph{The \fixttspace\token{variable} attribute}

A \BoundaryCondition has a required attribute \token{variable} of type
\primtype{SIdRef}.
This attribute must be the identifier of an existing \Species object.
\TODO{explain variable}


\paragraph{The \fixttspace\token{type} attribute}

A \BoundaryCondition has a required attribute \token{type} of type
\primtype{BoundaryConditionKind}.
\TODO{explain type}


\paragraph{The \fixttspace\token{coordinateBoundary} attribute}

A \BoundaryCondition has an optional attribute
\token{coordinateBoundary} of type \primtype{SIdRef}.
This attribute must be the identifier of an existing \Boundary object.
\TODO{explain coordinateBoundary}


\paragraph{The \fixttspace\token{boundaryDomainType} attribute}

A \BoundaryCondition has an optional attribute
\token{boundaryDomainType} of type \primtype{SIdRef}.
This attribute must be the identifier of an existing \DomainType object.
\TODO{explain boundaryDomainType}


% ---------------------------------------------------------
\subsection{The \class{DiffusionCoefficient} class}
\label{diffusioncoefficient-class}

% \begin{figure}[ht!]
%   \centering
%   \includegraphics[scale=0.6]{figures/spatial_diffusioncoefficient_uml.pdf}\\
% \caption{A UML representation of the \DiffusionCoefficient class for
% the \SpatialPackage. See \ref{conventions} for conventions related to
% this figure. }
%   \label{fig:spatial_diffusioncoefficient_uml}
% \end{figure}


\TODO{explain DiffusionCoefficient}

The \DiffusionCoefficient object derives from the \SBase class and thus
inherits any attributes and elements that are present on this class.
In addition the \DiffusionCoefficient object has the following
attributes.

\paragraph{The \fixttspace\token{variable} attribute}

A \DiffusionCoefficient has a required attribute \token{variable} of
type \primtype{SIdRef}.
This attribute must be the identifier of an existing \Species object.
\TODO{explain variable}


\paragraph{The \fixttspace\token{type} attribute}

A \DiffusionCoefficient has a required attribute \token{type} of type
\primtype{DiffusionKind}.
\TODO{explain type}


\paragraph{The \fixttspace\token{coordinateReferenceOne} attribute}

A \DiffusionCoefficient has an optional attribute
\token{coordinateReferenceOne} of type \primtype{CoordinateKind}.
\TODO{explain coordinateReferenceOne}


\paragraph{The \fixttspace\token{coordinateReferenceTwo} attribute}

A \DiffusionCoefficient has an optional attribute
\token{coordinateReferenceTwo} of type \primtype{CoordinateKind}.
\TODO{explain coordinateReferenceTwo}


% ---------------------------------------------------------
\subsection{The extended \class{Reaction} class}
\label{extended-reaction-class}

% \begin{figure}[ht!]
%   \centering
%   \includegraphics[scale=0.6]{figures/spatial_extended_reaction_uml.pdf}\\
% \caption{A UML representation of the extended \Reaction class for the
% \SpatialPackage. See \ref{conventions} for conventions related to
% this figure. }
%   \label{fig:spatial_extended_reaction_uml}
% \end{figure}


\TODO{explain where Reaction comes from}

The \SpatialProcessesPackage extends the \class{Reaction} object with
the addition of
the following attributes..

\paragraph{The \fixttspace\token{isLocal} attribute}

A \Reaction has a required attribute \token{isLocal} of type
\primtype{bool}.
\TODO{explain isLocal}


% ---------------------------------------------------------
\subsection{The \class{ListOfDomainTypes} class}
\label{listofdomaintypes-class}

\TODO{explain ListOfDomainTypes}

The \ListOfDomainTypes object derives from the \class{SBase} and
inherits the core attributes and subobjects from that class. It contains
zero or more objects of type \DomainType.

% ---------------------------------------------------------
\subsection{The \class{DomainType} class}
\label{domaintype-class}

% \begin{figure}[ht!]
%   \centering
%   \includegraphics[scale=0.6]{figures/spatial_domaintype_uml.pdf}\\
% \caption{A UML representation of the \DomainType class for the
% \SpatialPackage. See \ref{conventions} for conventions related to
% this figure. }
%   \label{fig:spatial_domaintype_uml}
% \end{figure}


\TODO{explain DomainType}

The \DomainType object derives from the \SBase class and thus inherits
any attributes and elements that are present on this class.
In addition the \DomainType object has the following attributes.

\paragraph{The \fixttspace\token{id} attribute}

A \DomainType has a required attribute \token{id} of type
\primtype{SId}.
\TODO{explain id}


\paragraph{The \fixttspace\token{spatialDimensions} attribute}

A \DomainType has a required attribute \token{spatialDimensions} of type
\primtype{int}.
\TODO{explain spatialDimensions}


% ---------------------------------------------------------
\subsection{The \class{ListOfDomains} class}
\label{listofdomains-class}

\TODO{explain ListOfDomains}

The \ListOfDomains object derives from the \class{SBase} and inherits
the core attributes and subobjects from that class. It contains zero or
more objects of type \Domain.

% ---------------------------------------------------------
\subsection{The \class{Domain} class}
\label{domain-class}

% \begin{figure}[ht!]
%   \centering
%   \includegraphics[scale=0.6]{figures/spatial_domain_uml.pdf}\\
% \caption{A UML representation of the \Domain class for the
% \SpatialPackage. See \ref{conventions} for conventions related to
% this figure. }
%   \label{fig:spatial_domain_uml}
% \end{figure}


\TODO{explain Domain}

The \Domain object derives from the \SBase class and thus inherits any
attributes and elements that are present on this class.
A \Domain contains exactly one \ListOfInteriorPoints element.
In addition the \Domain object has the following attributes.

\paragraph{The \fixttspace\token{id} attribute}

A \Domain has a required attribute \token{id} of type \primtype{SId}.
\TODO{explain id}


\paragraph{The \fixttspace\token{domainType} attribute}

A \Domain has a required attribute \token{domainType} of type
\primtype{SIdRef}.
This attribute must be the identifier of an existing \DomainType object.
\TODO{explain domainType}


% ---------------------------------------------------------
\subsection{The \class{ListOfInteriorPoints} class}
\label{listofinteriorpoints-class}

\TODO{explain ListOfInteriorPoints}

The \ListOfInteriorPoints object derives from the \class{SBase} and
inherits the core attributes and subobjects from that class. It contains
zero or more objects of type \InteriorPoint.

% ---------------------------------------------------------
\subsection{The \class{InteriorPoint} class}
\label{interiorpoint-class}

% \begin{figure}[ht!]
%   \centering
%   \includegraphics[scale=0.6]{figures/spatial_interiorpoint_uml.pdf}\\
% \caption{A UML representation of the \InteriorPoint class for the
% \SpatialPackage. See \ref{conventions} for conventions related to
% this figure. }
%   \label{fig:spatial_interiorpoint_uml}
% \end{figure}


\TODO{explain InteriorPoint}

The \InteriorPoint object derives from the \SBase class and thus
inherits any attributes and elements that are present on this class.
In addition the \InteriorPoint object has the following attributes.

\paragraph{The \fixttspace\token{coordOne} attribute}

An \InteriorPoint has a required attribute \token{coordOne} of type
\primtype{double}.
\TODO{explain coordOne}


\paragraph{The \fixttspace\token{coordTwo} attribute}

An \InteriorPoint has an optional attribute \token{coordTwo} of type
\primtype{double}.
\TODO{explain coordTwo}


\paragraph{The \fixttspace\token{coordThree} attribute}

An \InteriorPoint has an optional attribute \token{coordThree} of type
\primtype{double}.
\TODO{explain coordThree}


% ---------------------------------------------------------
\subsection{The \class{Boundary} class}
\label{boundary-class}

% \begin{figure}[ht!]
%   \centering
%   \includegraphics[scale=0.6]{figures/spatial_boundary_uml.pdf}\\
% \caption{A UML representation of the \Boundary class for the
% \SpatialPackage. See \ref{conventions} for conventions related to
% this figure. }
%   \label{fig:spatial_boundary_uml}
% \end{figure}


\TODO{explain Boundary}

The \Boundary object derives from the \SBase class and thus inherits any
attributes and elements that are present on this class.
In addition the \Boundary object has the following attributes.

\paragraph{The \fixttspace\token{id} attribute}

A \Boundary has a required attribute \token{id} of type \primtype{SId}.
\TODO{explain id}


\paragraph{The \fixttspace\token{value} attribute}

A \Boundary has a required attribute \token{value} of type
\primtype{double}.
\TODO{explain value}


% ---------------------------------------------------------
\subsection{The \class{ListOfAdjacentDomains} class}
\label{listofadjacentdomains-class}

\TODO{explain ListOfAdjacentDomains}

The \ListOfAdjacentDomains object derives from the \class{SBase} and
inherits the core attributes and subobjects from that class. It contains
zero or more objects of type \AdjacentDomains.

% ---------------------------------------------------------
\subsection{The \class{AdjacentDomains} class}
\label{adjacentdomains-class}

% \begin{figure}[ht!]
%   \centering
%   \includegraphics[scale=0.6]{figures/spatial_adjacentdomains_uml.pdf}\\
% \caption{A UML representation of the \AdjacentDomains class for the
% \SpatialPackage. See \ref{conventions} for conventions related to
% this figure. }
%   \label{fig:spatial_adjacentdomains_uml}
% \end{figure}


\TODO{explain AdjacentDomains}

The \AdjacentDomains object derives from the \SBase class and thus
inherits any attributes and elements that are present on this class.
In addition the \AdjacentDomains object has the following attributes.

\paragraph{The \fixttspace\token{id} attribute}

An \AdjacentDomains has a required attribute \token{id} of type
\primtype{SId}.
\TODO{explain id}


\paragraph{The \fixttspace\token{domainOne} attribute}

An \AdjacentDomains has a required attribute \token{domainOne} of type
\primtype{SIdRef}.
This attribute must be the identifier of an existing \Domain object.
\TODO{explain domainOne}


\paragraph{The \fixttspace\token{domainTwo} attribute}

An \AdjacentDomains has a required attribute \token{domainTwo} of type
\primtype{SIdRef}.
This attribute must be the identifier of an existing \Domain object.
\TODO{explain domainTwo}


% ---------------------------------------------------------
\subsection{The \class{ListOfGeometryDefinitions} class}
\label{listofgeometrydefinitions-class}

\TODO{explain ListOfGeometryDefinitions}

The \ListOfGeometryDefinitions object derives from the \class{SBase} and
inherits the core attributes and subobjects from that class. It contains
zero or more objects of type \GeometryDefinition.

% ---------------------------------------------------------
\subsection{The \class{GeometryDefinition} class}
\label{geometrydefinition-class}

% \begin{figure}[ht!]
%   \centering
%   \includegraphics[scale=0.6]{figures/spatial_geometrydefinition_uml.pdf}\\
% \caption{A UML representation of the \GeometryDefinition class for
% the \SpatialPackage. See \ref{conventions} for conventions related to
% this figure. }
%   \label{fig:spatial_geometrydefinition_uml}
% \end{figure}


\TODO{explain GeometryDefinition}

The \GeometryDefinition object derives from the \SBase class and thus
inherits any attributes and elements that are present on this class.
In addition the \GeometryDefinition object has the following attributes.

\paragraph{The \fixttspace\token{id} attribute}

A \GeometryDefinition has a required attribute \token{id} of type
\primtype{SId}.
\TODO{explain id}


\paragraph{The \fixttspace\token{isActive} attribute}

A \GeometryDefinition has a required attribute \token{isActive} of type
\primtype{bool}.
\TODO{explain isActive}


% ---------------------------------------------------------
\subsection{The \class{ListOfCoordinateComponents} class}
\label{listofcoordinatecomponents-class}

\TODO{explain ListOfCoordinateComponents}

The \ListOfCoordinateComponents object derives from the \class{SBase}
and inherits the core attributes and subobjects from that class. It
contains zero or more objects of type \CoordinateComponent.

% ---------------------------------------------------------
\subsection{The \class{CoordinateComponent} class}
\label{coordinatecomponent-class}

% \begin{figure}[ht!]
%   \centering
%   \includegraphics[scale=0.6]{figures/spatial_coordinatecomponent_uml.pdf}\\
% \caption{A UML representation of the \CoordinateComponent class for
% the \SpatialPackage. See \ref{conventions} for conventions related to
% this figure. }
%   \label{fig:spatial_coordinatecomponent_uml}
% \end{figure}


\TODO{explain CoordinateComponent}

The \CoordinateComponent object derives from the \SBase class and thus
inherits any attributes and elements that are present on this class.
A \CoordinateComponent contains at most one \Boundary element.
A \CoordinateComponent contains at most one \Boundary element.
In addition the \CoordinateComponent object has the following
attributes.

\paragraph{The \fixttspace\token{id} attribute}

A \CoordinateComponent has a required attribute \token{id} of type
\primtype{SId}.
\TODO{explain id}


\paragraph{The \fixttspace\token{type} attribute}

A \CoordinateComponent has a required attribute \token{type} of type
\primtype{CoordinateKind}.
\TODO{explain type}


\paragraph{The \fixttspace\token{unit} attribute}

A \CoordinateComponent has an optional attribute \token{unit} of type
\primtype{UnitSIdRef}.
\TODO{explain unit}


% ---------------------------------------------------------
\subsection{The \class{SampledFieldGeometry} class}
\label{sampledfieldgeometry-class}

% \begin{figure}[ht!]
%   \centering
%   \includegraphics[scale=0.6]{figures/spatial_sampledfieldgeometry_uml.pdf}\\
% \caption{A UML representation of the \SampledFieldGeometry class for
% the \SpatialPackage. See \ref{conventions} for conventions related to
% this figure. }
%   \label{fig:spatial_sampledfieldgeometry_uml}
% \end{figure}


\TODO{explain SampledFieldGeometry}

The \SampledFieldGeometry object derives from the \GeometryDefinition
class and thus inherits any attributes and elements that are present on
this class.
A \SampledFieldGeometry contains exactly one \ListOfSampledVolumes
element.
In addition the \SampledFieldGeometry object has the following
attributes.

\paragraph{The \fixttspace\token{sampledField} attribute}

A \SampledFieldGeometry has a required attribute \token{sampledField} of
type \primtype{SIdRef}.
This attribute must be the identifier of an existing \SampledField
object.
\TODO{explain sampledField}


% ---------------------------------------------------------
\subsection{The \class{ListOfSampledFields} class}
\label{listofsampledfields-class}

\TODO{explain ListOfSampledFields}

The \ListOfSampledFields object derives from the \class{SBase} and
inherits the core attributes and subobjects from that class. It contains
zero or more objects of type \SampledField.

% ---------------------------------------------------------
\subsection{The \class{SampledField} class}
\label{sampledfield-class}

% \begin{figure}[ht!]
%   \centering
%   \includegraphics[scale=0.6]{figures/spatial_sampledfield_uml.pdf}\\
% \caption{A UML representation of the \SampledField class for the
% \SpatialPackage. See \ref{conventions} for conventions related to
% this figure. }
%   \label{fig:spatial_sampledfield_uml}
% \end{figure}


\TODO{explain SampledField}

The \SampledField object derives from the \SBase class and thus inherits
any attributes and elements that are present on this class.
In addition the \SampledField object has the following attributes.

\paragraph{The \fixttspace\token{id} attribute}

A \SampledField has a required attribute \token{id} of type
\primtype{SId}.
\TODO{explain id}


\paragraph{The \fixttspace\token{dataType} attribute}

A \SampledField has a required attribute \token{dataType} of type
\primtype{DataKind}.
\TODO{explain dataType}


\paragraph{The \fixttspace\token{numSamplesOne} attribute}

A \SampledField has a required attribute \token{numSamplesOne} of type
\primtype{int}.
\TODO{explain numSamplesOne}


\paragraph{The \fixttspace\token{numSamplesTwo} attribute}

A \SampledField has an optional attribute \token{numSamplesTwo} of type
\primtype{int}.
\TODO{explain numSamplesTwo}


\paragraph{The \fixttspace\token{numSamplesThree} attribute}

A \SampledField has an optional attribute \token{numSamplesThree} of
type \primtype{int}.
\TODO{explain numSamplesThree}


\paragraph{The \fixttspace\token{interpolationType} attribute}

A \SampledField has a required attribute \token{interpolationType} of
type \primtype{InterpolationKind}.
\TODO{explain interpolationType}


\paragraph{The \fixttspace\token{compression} attribute}

A \SampledField has a required attribute \token{compression} of type
\primtype{CompressionKind}.
\TODO{explain compression}


\paragraph{The \fixttspace\token{samples} attribute}

A \SampledField has a required attribute \token{samples} consisting of
an array of \primtype{Integer}.
\TODO{explain samples}


\paragraph{The \fixttspace\token{samplesLength} attribute}

A \SampledField has a required attribute \token{samplesLength} of type
\primtype{int}.
\TODO{explain samplesLength}


% ---------------------------------------------------------
\subsection{The \class{ListOfSampledVolumes} class}
\label{listofsampledvolumes-class}

\TODO{explain ListOfSampledVolumes}

The \ListOfSampledVolumes object derives from the \class{SBase} and
inherits the core attributes and subobjects from that class. It contains
zero or more objects of type \SampledVolume.

% ---------------------------------------------------------
\subsection{The \class{SampledVolume} class}
\label{sampledvolume-class}

% \begin{figure}[ht!]
%   \centering
%   \includegraphics[scale=0.6]{figures/spatial_sampledvolume_uml.pdf}\\
% \caption{A UML representation of the \SampledVolume class for the
% \SpatialPackage. See \ref{conventions} for conventions related to
% this figure. }
%   \label{fig:spatial_sampledvolume_uml}
% \end{figure}


\TODO{explain SampledVolume}

The \SampledVolume object derives from the \SBase class and thus
inherits any attributes and elements that are present on this class.
In addition the \SampledVolume object has the following attributes.

\paragraph{The \fixttspace\token{id} attribute}

A \SampledVolume has a required attribute \token{id} of type
\primtype{SId}.
\TODO{explain id}


\paragraph{The \fixttspace\token{domainType} attribute}

A \SampledVolume has a required attribute \token{domainType} of type
\primtype{SIdRef}.
This attribute must be the identifier of an existing \DomainType object.
\TODO{explain domainType}


\paragraph{The \fixttspace\token{sampledValue} attribute}

A \SampledVolume has a required attribute \token{sampledValue} of type
\primtype{double}.
\TODO{explain sampledValue}


\paragraph{The \fixttspace\token{minValue} attribute}

A \SampledVolume has an optional attribute \token{minValue} of type
\primtype{double}.
\TODO{explain minValue}


\paragraph{The \fixttspace\token{maxValue} attribute}

A \SampledVolume has an optional attribute \token{maxValue} of type
\primtype{double}.
\TODO{explain maxValue}


% ---------------------------------------------------------
\subsection{The \class{AnalyticGeometry} class}
\label{analyticgeometry-class}

% \begin{figure}[ht!]
%   \centering
%   \includegraphics[scale=0.6]{figures/spatial_analyticgeometry_uml.pdf}\\
% \caption{A UML representation of the \AnalyticGeometry class for the
% \SpatialPackage. See \ref{conventions} for conventions related to
% this figure. }
%   \label{fig:spatial_analyticgeometry_uml}
% \end{figure}


\TODO{explain AnalyticGeometry}

The \AnalyticGeometry object derives from the \GeometryDefinition class
and thus inherits any attributes and elements that are present on this
class.
An \AnalyticGeometry contains exactly one \ListOfAnalyticVolumes
element.
% ---------------------------------------------------------
\subsection{The \class{ListOfAnalyticVolumes} class}
\label{listofanalyticvolumes-class}

\TODO{explain ListOfAnalyticVolumes}

The \ListOfAnalyticVolumes object derives from the \class{SBase} and
inherits the core attributes and subobjects from that class. It contains
zero or more objects of type \AnalyticVolume.

% ---------------------------------------------------------
\subsection{The \class{AnalyticVolume} class}
\label{analyticvolume-class}

% \begin{figure}[ht!]
%   \centering
%   \includegraphics[scale=0.6]{figures/spatial_analyticvolume_uml.pdf}\\
% \caption{A UML representation of the \AnalyticVolume class for the
% \SpatialPackage. See \ref{conventions} for conventions related to
% this figure. }
%   \label{fig:spatial_analyticvolume_uml}
% \end{figure}


\TODO{explain AnalyticVolume}

The \AnalyticVolume object derives from the \SBase class and thus
inherits any attributes and elements that are present on this class.
An \AnalyticVolume contains exactly one \class{ASTNode} element.
In addition the \AnalyticVolume object has the following attributes.

\paragraph{The \fixttspace\token{id} attribute}

An \AnalyticVolume has a required attribute \token{id} of type
\primtype{SId}.
\TODO{explain id}


\paragraph{The \fixttspace\token{functionType} attribute}

An \AnalyticVolume has a required attribute \token{functionType} of type
\primtype{FunctionKind}.
\TODO{explain functionType}


\paragraph{The \fixttspace\token{ordinal} attribute}

An \AnalyticVolume has an optional attribute \token{ordinal} of type
\primtype{int}.
\TODO{explain ordinal}


\paragraph{The \fixttspace\token{domainType} attribute}

An \AnalyticVolume has a required attribute \token{domainType} of type
\primtype{SIdRef}.
This attribute must be the identifier of an existing \DomainType object.
\TODO{explain domainType}


% ---------------------------------------------------------
\subsection{The \class{ParametricGeometry} class}
\label{parametricgeometry-class}

% \begin{figure}[ht!]
%   \centering
%   \includegraphics[scale=0.6]{figures/spatial_parametricgeometry_uml.pdf}\\
% \caption{A UML representation of the \ParametricGeometry class for
% the \SpatialPackage. See \ref{conventions} for conventions related to
% this figure. }
%   \label{fig:spatial_parametricgeometry_uml}
% \end{figure}


\TODO{explain ParametricGeometry}

The \ParametricGeometry object derives from the \GeometryDefinition
class and thus inherits any attributes and elements that are present on
this class.
A \ParametricGeometry contains exactly one \SpatialPoints element.
A \ParametricGeometry contains exactly one \ListOfParametricObjects
element.
% ---------------------------------------------------------
\subsection{The \class{ListOfParametricObjects} class}
\label{listofparametricobjects-class}

\TODO{explain ListOfParametricObjects}

The \ListOfParametricObjects object derives from the \class{SBase} and
inherits the core attributes and subobjects from that class. It contains
zero or more objects of type \ParametricObject.

% ---------------------------------------------------------
\subsection{The \class{ParametricObject} class}
\label{parametricobject-class}

% \begin{figure}[ht!]
%   \centering
%   \includegraphics[scale=0.6]{figures/spatial_parametricobject_uml.pdf}\\
% \caption{A UML representation of the \ParametricObject class for the
% \SpatialPackage. See \ref{conventions} for conventions related to
% this figure. }
%   \label{fig:spatial_parametricobject_uml}
% \end{figure}


\TODO{explain ParametricObject}

The \ParametricObject object derives from the \SBase class and thus
inherits any attributes and elements that are present on this class.
In addition the \ParametricObject object has the following attributes.

\paragraph{The \fixttspace\token{id} attribute}

A \ParametricObject has a required attribute \token{id} of type
\primtype{SId}.
\TODO{explain id}


\paragraph{The \fixttspace\token{polygonType} attribute}

A \ParametricObject has a required attribute \token{polygonType} of type
\primtype{PolygonKind}.
\TODO{explain polygonType}


\paragraph{The \fixttspace\token{domainType} attribute}

A \ParametricObject has a required attribute \token{domainType} of type
\primtype{SIdRef}.
This attribute must be the identifier of an existing \DomainType object.
\TODO{explain domainType}


\paragraph{The \fixttspace\token{pointIndex} attribute}

A \ParametricObject has a required attribute \token{pointIndex}
consisting of an array of \primtype{Integer}.
\TODO{explain pointIndex}


\paragraph{The \fixttspace\token{pointIndexLength} attribute}

A \ParametricObject has a required attribute \token{pointIndexLength} of
type \primtype{int}.
\TODO{explain pointIndexLength}


\paragraph{The \fixttspace\token{compression} attribute}

A \ParametricObject has a required attribute \token{compression} of type
\primtype{CompressionKind}.
\TODO{explain compression}


\paragraph{The \fixttspace\token{dataType} attribute}

A \ParametricObject has an optional attribute \token{dataType} of type
\primtype{DataKind}.
\TODO{explain dataType}


% ---------------------------------------------------------
\subsection{The \class{CSGeometry} class}
\label{csgeometry-class}

% \begin{figure}[ht!]
%   \centering
%   \includegraphics[scale=0.6]{figures/spatial_csgeometry_uml.pdf}\\
% \caption{A UML representation of the \CSGeometry class for the
% \SpatialPackage. See \ref{conventions} for conventions related to
% this figure. }
%   \label{fig:spatial_csgeometry_uml}
% \end{figure}


\TODO{explain CSGeometry}

The \CSGeometry object derives from the \GeometryDefinition class and
thus inherits any attributes and elements that are present on this
class.
A \CSGeometry contains exactly one \ListOfCSGObjects element.
% ---------------------------------------------------------
\subsection{The \class{ListOfCSGObjects} class}
\label{listofcsgobjects-class}

\TODO{explain ListOfCSGObjects}

The \ListOfCSGObjects object derives from the \class{SBase} and inherits
the core attributes and subobjects from that class. It contains zero or
more objects of type \CSGObject.

% ---------------------------------------------------------
\subsection{The \class{CSGObject} class}
\label{csgobject-class}

% \begin{figure}[ht!]
%   \centering
%   \includegraphics[scale=0.6]{figures/spatial_csgobject_uml.pdf}\\
% \caption{A UML representation of the \CSGObject class for the
% \SpatialPackage. See \ref{conventions} for conventions related to
% this figure. }
%   \label{fig:spatial_csgobject_uml}
% \end{figure}


\TODO{explain CSGObject}

The \CSGObject object derives from the \SBase class and thus inherits
any attributes and elements that are present on this class.
A \CSGObject contains at most one \CSGNode element.
In addition the \CSGObject object has the following attributes.

\paragraph{The \fixttspace\token{id} attribute}

A \CSGObject has a required attribute \token{id} of type \primtype{SId}.
\TODO{explain id}


\paragraph{The \fixttspace\token{domainType} attribute}

A \CSGObject has a required attribute \token{domainType} of type
\primtype{SIdRef}.
This attribute must be the identifier of an existing \DomainType object.
\TODO{explain domainType}


\paragraph{The \fixttspace\token{ordinal} attribute}

A \CSGObject has an optional attribute \token{ordinal} of type
\primtype{int}.
\TODO{explain ordinal}


% ---------------------------------------------------------
\subsection{The \class{ListOfCSGNodes} class}
\label{listofcsgnodes-class}

\TODO{explain ListOfCSGNodes}

The \ListOfCSGNodes object derives from the \class{SBase} and inherits
the core attributes and subobjects from that class. It contains zero or
more objects of type \CSGNode.

% ---------------------------------------------------------
\subsection{The \class{CSGNode} class}
\label{csgnode-class}

% \begin{figure}[ht!]
%   \centering
%   \includegraphics[scale=0.6]{figures/spatial_csgnode_uml.pdf}\\
% \caption{A UML representation of the \CSGNode class for the
% \SpatialPackage. See \ref{conventions} for conventions related to
% this figure. }
%   \label{fig:spatial_csgnode_uml}
% \end{figure}


\TODO{explain CSGNode}

The \CSGNode object derives from the \SBase class and thus inherits any
attributes and elements that are present on this class.
In addition the \CSGNode object has the following attributes.

\paragraph{The \fixttspace\token{id} attribute}

A \CSGNode has an optional attribute \token{id} of type \primtype{SId}.
\TODO{explain id}


% ---------------------------------------------------------
\subsection{The \class{CSGTransformation} class}
\label{csgtransformation-class}

% \begin{figure}[ht!]
%   \centering
%   \includegraphics[scale=0.6]{figures/spatial_csgtransformation_uml.pdf}\\
% \caption{A UML representation of the \CSGTransformation class for the
% \SpatialPackage. See \ref{conventions} for conventions related to
% this figure. }
%   \label{fig:spatial_csgtransformation_uml}
% \end{figure}


\TODO{explain CSGTransformation}

The \CSGTransformation object derives from the \CSGNode class and thus
inherits any attributes and elements that are present on this class.
A \CSGTransformation contains exactly one \CSGNode element.
% ---------------------------------------------------------
\subsection{The \class{CSGTranslation} class}
\label{csgtranslation-class}

% \begin{figure}[ht!]
%   \centering
%   \includegraphics[scale=0.6]{figures/spatial_csgtranslation_uml.pdf}\\
% \caption{A UML representation of the \CSGTranslation class for the
% \SpatialPackage. See \ref{conventions} for conventions related to
% this figure. }
%   \label{fig:spatial_csgtranslation_uml}
% \end{figure}


\TODO{explain CSGTranslation}

The \CSGTranslation object derives from the \CSGTransformation class and
thus inherits any attributes and elements that are present on this
class.
In addition the \CSGTranslation object has the following attributes.

\paragraph{The \fixttspace\token{translateX} attribute}

A \CSGTranslation has a required attribute \token{translateX} of type
\primtype{double}.
\TODO{explain translateX}


\paragraph{The \fixttspace\token{translateY} attribute}

A \CSGTranslation has an optional attribute \token{translateY} of type
\primtype{double}.
\TODO{explain translateY}


\paragraph{The \fixttspace\token{translateZ} attribute}

A \CSGTranslation has an optional attribute \token{translateZ} of type
\primtype{double}.
\TODO{explain translateZ}


% ---------------------------------------------------------
\subsection{The \class{CSGRotation} class}
\label{csgrotation-class}

% \begin{figure}[ht!]
%   \centering
%   \includegraphics[scale=0.6]{figures/spatial_csgrotation_uml.pdf}\\
% \caption{A UML representation of the \CSGRotation class for the
% \SpatialPackage. See \ref{conventions} for conventions related to
% this figure. }
%   \label{fig:spatial_csgrotation_uml}
% \end{figure}


\TODO{explain CSGRotation}

The \CSGRotation object derives from the \CSGTransformation class and
thus inherits any attributes and elements that are present on this
class.
In addition the \CSGRotation object has the following attributes.

\paragraph{The \fixttspace\token{rotateX} attribute}

A \CSGRotation has a required attribute \token{rotateX} of type
\primtype{double}.
\TODO{explain rotateX}


\paragraph{The \fixttspace\token{rotateY} attribute}

A \CSGRotation has an optional attribute \token{rotateY} of type
\primtype{double}.
\TODO{explain rotateY}


\paragraph{The \fixttspace\token{rotateZ} attribute}

A \CSGRotation has an optional attribute \token{rotateZ} of type
\primtype{double}.
\TODO{explain rotateZ}


\paragraph{The \fixttspace\token{rotateAngleInRadians} attribute}

A \CSGRotation has a required attribute \token{rotateAngleInRadians} of
type \primtype{double}.
\TODO{explain rotateAngleInRadians}


% ---------------------------------------------------------
\subsection{The \class{CSGScale} class}
\label{csgscale-class}

% \begin{figure}[ht!]
%   \centering
%   \includegraphics[scale=0.6]{figures/spatial_csgscale_uml.pdf}\\
% \caption{A UML representation of the \CSGScale class for the
% \SpatialPackage. See \ref{conventions} for conventions related to
% this figure. }
%   \label{fig:spatial_csgscale_uml}
% \end{figure}


\TODO{explain CSGScale}

The \CSGScale object derives from the \CSGTransformation class and thus
inherits any attributes and elements that are present on this class.
In addition the \CSGScale object has the following attributes.

\paragraph{The \fixttspace\token{scaleX} attribute}

A \CSGScale has a required attribute \token{scaleX} of type
\primtype{double}.
\TODO{explain scaleX}


\paragraph{The \fixttspace\token{scaleY} attribute}

A \CSGScale has an optional attribute \token{scaleY} of type
\primtype{double}.
\TODO{explain scaleY}


\paragraph{The \fixttspace\token{scaleZ} attribute}

A \CSGScale has an optional attribute \token{scaleZ} of type
\primtype{double}.
\TODO{explain scaleZ}


% ---------------------------------------------------------
\subsection{The \class{CSGHomogeneousTransformation} class}
\label{csghomogeneoustransformation-class}

% \begin{figure}[ht!]
%   \centering
%   \includegraphics[scale=0.6]{figures/spatial_csghomogeneoustransformation_uml.pdf}\\
% \caption{A UML representation of the \CSGHomogeneousTransformation
% class for the \SpatialPackage. See \ref{conventions} for conventions
% related to this figure. }
%   \label{fig:spatial_csghomogeneoustransformation_uml}
% \end{figure}


\TODO{explain CSGHomogeneousTransformation}

The \CSGHomogeneousTransformation object derives from the
\CSGTransformation class and thus inherits any attributes and elements
that are present on this class.
A \CSGHomogeneousTransformation contains at most one
\TransformationComponent element.
A \CSGHomogeneousTransformation contains at most one
\TransformationComponent element.
% ---------------------------------------------------------
\subsection{The \class{TransformationComponent} class}
\label{transformationcomponent-class}

% \begin{figure}[ht!]
%   \centering
%   \includegraphics[scale=0.6]{figures/spatial_transformationcomponent_uml.pdf}\\
% \caption{A UML representation of the \TransformationComponent class
% for the \SpatialPackage. See \ref{conventions} for conventions
% related to this figure. }
%   \label{fig:spatial_transformationcomponent_uml}
% \end{figure}


\TODO{explain TransformationComponent}

The \TransformationComponent object derives from the \SBase class and
thus inherits any attributes and elements that are present on this
class.
In addition the \TransformationComponent object has the following
attributes.

\paragraph{The \fixttspace\token{components} attribute}

A \TransformationComponent has a required attribute \token{components}
consisting of an array of \primtype{Double}.
\TODO{explain components}


\paragraph{The \fixttspace\token{componentsLength} attribute}

A \TransformationComponent has a required attribute
\token{componentsLength} of type \primtype{int}.
\TODO{explain componentsLength}


% ---------------------------------------------------------
\subsection{The \class{CSGPrimitive} class}
\label{csgprimitive-class}

% \begin{figure}[ht!]
%   \centering
%   \includegraphics[scale=0.6]{figures/spatial_csgprimitive_uml.pdf}\\
% \caption{A UML representation of the \CSGPrimitive class for the
% \SpatialPackage. See \ref{conventions} for conventions related to
% this figure. }
%   \label{fig:spatial_csgprimitive_uml}
% \end{figure}


\TODO{explain CSGPrimitive}

The \CSGPrimitive object derives from the \CSGNode class and thus
inherits any attributes and elements that are present on this class.
In addition the \CSGPrimitive object has the following attributes.

\paragraph{The \fixttspace\token{primitiveType} attribute}

A \CSGPrimitive has a required attribute \token{primitiveType} of type
\primtype{PrimitiveKind}.
\TODO{explain primitiveType}


% ---------------------------------------------------------
\subsection{The \class{CSGSetOperator} class}
\label{csgsetoperator-class}

% \begin{figure}[ht!]
%   \centering
%   \includegraphics[scale=0.6]{figures/spatial_csgsetoperator_uml.pdf}\\
% \caption{A UML representation of the \CSGSetOperator class for the
% \SpatialPackage. See \ref{conventions} for conventions related to
% this figure. }
%   \label{fig:spatial_csgsetoperator_uml}
% \end{figure}


\TODO{explain CSGSetOperator}

The \CSGSetOperator object derives from the \CSGNode class and thus
inherits any attributes and elements that are present on this class.
A \CSGSetOperator contains exactly one \ListOfCSGNodes element.
In addition the \CSGSetOperator object has the following attributes.

\paragraph{The \fixttspace\token{operationType} attribute}

A \CSGSetOperator has a required attribute \token{operationType} of type
\primtype{SetOperation}.
\TODO{explain operationType}


\paragraph{The \fixttspace\token{complementA} attribute}

A \CSGSetOperator has an optional attribute \token{complementA} of type
\primtype{SIdRef}.
This attribute must be the identifier of an existing \CSGNode object.
\TODO{explain complementA}


\paragraph{The \fixttspace\token{complementB} attribute}

A \CSGSetOperator has an optional attribute \token{complementB} of type
\primtype{SIdRef}.
This attribute must be the identifier of an existing \CSGNode object.
\TODO{explain complementB}


% ---------------------------------------------------------
\subsection{The \class{MixedGeometry} class}
\label{mixedgeometry-class}

% \begin{figure}[ht!]
%   \centering
%   \includegraphics[scale=0.6]{figures/spatial_mixedgeometry_uml.pdf}\\
% \caption{A UML representation of the \MixedGeometry class for the
% \SpatialPackage. See \ref{conventions} for conventions related to
% this figure. }
%   \label{fig:spatial_mixedgeometry_uml}
% \end{figure}


\TODO{explain MixedGeometry}

The \MixedGeometry object derives from the \GeometryDefinition class and
thus inherits any attributes and elements that are present on this
class.
A \MixedGeometry contains exactly one \ListOfGeometryDefinitions
element.
A \MixedGeometry contains exactly one \ListOfOrdinalMappings element.
% ---------------------------------------------------------
\subsection{The \class{ListOfOrdinalMappings} class}
\label{listofordinalmappings-class}

\TODO{explain ListOfOrdinalMappings}

The \ListOfOrdinalMappings object derives from the \class{SBase} and
inherits the core attributes and subobjects from that class. It contains
zero or more objects of type \OrdinalMapping.

% ---------------------------------------------------------
\subsection{The \class{OrdinalMapping} class}
\label{ordinalmapping-class}

% \begin{figure}[ht!]
%   \centering
%   \includegraphics[scale=0.6]{figures/spatial_ordinalmapping_uml.pdf}\\
% \caption{A UML representation of the \OrdinalMapping class for the
% \SpatialPackage. See \ref{conventions} for conventions related to
% this figure. }
%   \label{fig:spatial_ordinalmapping_uml}
% \end{figure}


\TODO{explain OrdinalMapping}

The \OrdinalMapping object derives from the \SBase class and thus
inherits any attributes and elements that are present on this class.
In addition the \OrdinalMapping object has the following attributes.

\paragraph{The \fixttspace\token{geometryDefinition} attribute}

An \OrdinalMapping has a required attribute \token{geometryDefinition}
of type \primtype{SIdRef}.
This attribute must be the identifier of an existing \GeometryDefinition
object.
\TODO{explain geometryDefinition}


\paragraph{The \fixttspace\token{ordinal} attribute}

An \OrdinalMapping has a required attribute \token{ordinal} of type
\primtype{int}.
\TODO{explain ordinal}


% ---------------------------------------------------------
\subsection{The \class{SpatialPoints} class}
\label{spatialpoints-class}

% \begin{figure}[ht!]
%   \centering
%   \includegraphics[scale=0.6]{figures/spatial_spatialpoints_uml.pdf}\\
% \caption{A UML representation of the \SpatialPoints class for the
% \SpatialPackage. See \ref{conventions} for conventions related to
% this figure. }
%   \label{fig:spatial_spatialpoints_uml}
% \end{figure}


\TODO{explain SpatialPoints}

The \SpatialPoints object derives from the \SBase class and thus
inherits any attributes and elements that are present on this class.
In addition the \SpatialPoints object has the following attributes.

\paragraph{The \fixttspace\token{id} attribute}

A \SpatialPoints has a required attribute \token{id} of type
\primtype{SId}.
\TODO{explain id}


\paragraph{The \fixttspace\token{compression} attribute}

A \SpatialPoints has a required attribute \token{compression} of type
\primtype{CompressionKind}.
\TODO{explain compression}


\paragraph{The \fixttspace\token{arrayData} attribute}

A \SpatialPoints has a required attribute \token{arrayData} consisting
of an array of \primtype{Double}.
\TODO{explain arrayData}


\paragraph{The \fixttspace\token{arrayDataLength} attribute}

A \SpatialPoints has a required attribute \token{arrayDataLength} of
type \primtype{int}.
\TODO{explain arrayDataLength}


\paragraph{The \fixttspace\token{dataType} attribute}

A \SpatialPoints has an optional attribute \token{dataType} of type
\primtype{DataKind}.
\TODO{explain dataType}


